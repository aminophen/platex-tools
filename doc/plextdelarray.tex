%# -*- coding: utf-8 -*-
\ifdefined\epTeXinputencoding % defined in e-pTeX (> TL2016)
  \epTeXinputencoding utf8    % ensure utf-8 encoding for platex
\fi

\documentclass[a4paper]{jsarticle}
\usepackage{doc}
\usepackage{delarray}
\usepackage{plextdelarray}
\GetFileInfo{plextdelarray.sty}
\title{Package \textsf{plextdelarray} \fileversion}
\author{Hironobu Yamashita}
\date{\filedate}
\begin{document}

\maketitle

Package \textsf{delarray}, included in \textsf{latex-tools} bundle, is
incompatible with \textsf{plext} package for p\LaTeX.
The package \textsf{plextarray} resolves this incompatibility,
and enables coexistence of both extensions provided by \textsf{plext}
and \textsf{delarray}. If you are not familiar with Japanese p\LaTeX\
and \textsf{plext} package, please find the documentation of
\textsf{plextarray} for brief explanation.

This package is part of \textsf{platex-tools} bundle:
\begin{verbatim}
  https://github.com/aminophen/platex-tools
\end{verbatim}

The rest of this document is written in Japanese, and includes some
examples of usage.

\bigskip

\textsf{latex-tools}バンドルに含まれる\textsf{delarray}パッケージは、
p\LaTeX の拡張パッケージである\textsf{plext}パッケージと互換性がありません。
この\textsf{plextdelarray}パッケージは、\textsf{plext}パッケージの組方向
オプション拡張と\textsf{delarray}パッケージの拡張の両方を「一応使える」状態
にするものです。ただし、\textsf{plext}の内部実装の都合上、\textsf{delarray}と
完全に同じ結果にならない場合があるかもしれません\footnote{本パッケージは
「\textsf{delarray}を読み込んだだけで\textsf{plext}の部分的な縦書き機能が
すべて使えなくなってしまって困った」という場合の、あくまで非常手段という
程度にとらえてください。}。

\section{使いかた}

\textsf{plext}と\textsf{delarray}を共存させたいときに、プリアンブルに
\verb+\usepackage{plextdelarray}+と書きます。既にいずれかのパッケージが
読み込まれていても問題ありません。また、\textsf{plextdelarray}パッケージを
読み込めば、\textsf{plext}と\textsf{delarray}も自動的に読み込まれます。
また、\textsf{plextarray}パッケージも読み込みます。

以下に例を示します。左側が\textsf{delarray}の機能で左右括弧を付けたもの、
右側が通常の\textsf{array}の機能で左右括弧を付けたものです。
特に\verb+[t]+と\verb+[b]+は、\textsf{delarray}なしでは実現できない
ようです(例は\textsf{delarray}のドキュメントを少し改変)。

\bigskip
\begin{minipage}{0.5\linewidth}
\begin{verbatim}
  \documentclass{jsarticle}
  %\usepackage{plext}
  %\usepackage{delarray}
  \usepackage{plextdelarray}
  \begin{document}
  \[
    \begin{array}[t]\{{c}\}
      3 \\ 4 \\ 5  \end{array}
    \begin{array}[c]\{{c}\}
      2 \\ 3 \\ 4  \end{array}
    \begin{array}[b]\{{c}\}
      1 \\ 2 \\ 3  \end{array}
  \]
  \[
    \left\{  \begin{array}[t]{c}
      3 \\ 4 \\ 5  \end{array}  \right\}
    \left\{  \begin{array}[c]{c}
      2 \\ 3 \\ 4  \end{array}  \right\}
    \left\{  \begin{array}[b]{c}
      1 \\ 2 \\ 3  \end{array}  \right\}
  \]
  \end{document}
\end{verbatim}
\end{minipage}
\begin{minipage}{0.4\linewidth}
\mbox{}\\[10ex]
  \[
    \begin{array}[t]\{{c}\}
      3 \\ 4 \\ 5  \end{array}
    \begin{array}[c]\{{c}\}
      2 \\ 3 \\ 4  \end{array}
    \begin{array}[b]\{{c}\}
      1 \\ 2 \\ 3  \end{array}
  \]
\\[5ex]
  \[
    \left\{  \begin{array}[t]{c}
      3 \\ 4 \\ 5  \end{array}  \right\}
    \left\{  \begin{array}[c]{c}
      2 \\ 3 \\ 4  \end{array}  \right\}
    \left\{  \begin{array}[b]{c}
      1 \\ 2 \\ 3  \end{array}  \right\}
  \]
\end{minipage}
\bigskip

\textsf{plext}と\textsf{delarray}の両方の拡張を同じ箇所で使用することも
できなくはありませんが、役に立つ状況は少ないと思います。

\bigskip
\begin{minipage}{0.5\linewidth}
\begin{verbatim}
  \documentclass{jsarticle}
  %\usepackage{plext}
  %\usepackage{delarray}
  \usepackage{plextdelarray}
  \begin{document}
  \[
    \begin{array}<t>[t]\{{c}\}
      3 \\ 4 \\ 5  \end{array}
    \begin{array}<t>[c]\{{c}\}
      2 \\ 3 \\ 4  \end{array}
    \begin{array}<t>[b]\{{c}\}
      1 \\ 2 \\ 3  \end{array}
  \]
  \[
  \newcolumntype{L}{>{$}l<{$}}
  f(x)=
    \begin{array}<t>\{{lL}.
      0 & if $x=0$ \\
      \sin(x)/x & otherwise
    \end{array}
  \]
  \end{document}
\end{verbatim}
\end{minipage}
\begin{minipage}{0.4\linewidth}
\mbox{}\\[12ex]
  \[
    \begin{array}<t>[t]\{{c}\}
      3 \\ 4 \\ 5  \end{array}
    \begin{array}<t>[c]\{{c}\}
      2 \\ 3 \\ 4  \end{array}
    \begin{array}<t>[b]\{{c}\}
      1 \\ 2 \\ 3  \end{array}
  \]
\\[6ex]
  \[
  \newcolumntype{L}{>{$}l<{$}}
  f(x)=
    \begin{array}<t>\{{lL}.
      0 & if $x=0$ \\
      \sin(x)/x & otherwise
    \end{array}
  \]
\end{minipage}

\end{document}
