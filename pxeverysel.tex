%# -*- coding: utf-8 -*-
\ifx\epTeXinputencoding\undefined\else % defined in e-pTeX (> TL2016)
  \epTeXinputencoding utf8    % ensure utf-8 encoding for platex
\fi

\documentclass[a4paper]{jsarticle}
\usepackage[nohyperref]{doc}% doc v3
\usepackage{pxeverysel}
\GetFileInfo{pxeverysel.sty}
\title{Package \textsf{pxeverysel} \fileversion}
\author{Takayuki Yato \& Hironobu Yamashita}
\date{\filedate}
\begin{document}

\maketitle

Package \textsf{everysel}, written by Martin Schr\"oder,
is incompatible with the definition of \verb+\selectfont+ redefined
by p\LaTeX\ kernel. For this reason, when using \textsf{everysel},
font selection scheme for Japanese characters goes wrong.
The package \textsf{pxeverysel} provides a patch for \textsf{everysel}
to work with p\LaTeX\ font selection.

This package is part of \textsf{platex-tools} bundle:
\begin{verbatim}
  https://github.com/aminophen/platex-tools
\end{verbatim}

\bigskip

Martin Schr\"oder氏による\textsf{everysel}パッケージをp\LaTeX で
使用すると、日本語の文字サイズが変わらなくなったり、
\textsf{otf}パッケージで文字化けが起こったりします。
これは、\pLaTeX カーネルが日本語用に再定義している\verb+\selectfont+が、
\textsf{everysel}によって書き換えられてしまうためです。
この\textsf{pxeverysel}パッケージを読み込むことで、p\LaTeX でも
\textsf{everysel}パッケージを使えるようになります。

\LaTeXe~2021-06-01以降では、\textsf{everysel}と同等の機能がカーネルに
実装されていて、その対処は\pLaTeXe{}カーネルでなされます。
この場合、\textsf{pxeverysel}パッケージは何もしません。

\newpage
\section{使いかた}

使いかたは、\textsf{everysel}パッケージの代わりに、あるいは
\textsf{everysel}パッケージに追加して、\textsf{pxeverysel}パッケージを
読み込むだけです\footnote{稀に\\\texttt{%
!~Package pxeverysel Error:~Patch too late!\\
(pxeverysel)~~~~~~~~~~~~~~~~Load pxeverysel earlier.%
}\\というエラーが出ることがあります。この場合は、
\textsf{pxeverysel}パッケージを少し早めに読み込んでみてください。}。
\textsf{everysel}を内部で読み込んでいる\textsf{ragged2e}を例に示します:
\begin{verbatim}
  \documentclass[a4paper]{jsarticle}
  \usepackage{ragged2e}% 読み込むだけでアウト
  \usepackage{pxeverysel}
  \begin{document}
  % 和文のフォントサイズが変わらない!
  {\TeX}はアレ{\Large{\TeX}はアレ}
  \end{document}
\end{verbatim}

\section*{References}

\begin{itemize}
\item vwcol.styとutf.styを併用したい\\
  \texttt{http://oku.edu.mie-u.ac.jp/tex/mod/forum/discuss.php?d=1763}
\item pLaTeXでeveryselしたい話\\
  \texttt{http://d.hatena.ne.jp/zrbabbler/20151212/1449898508}
\end{itemize}

\end{document}
