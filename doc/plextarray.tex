%# -*- coding: utf-8 -*-
\ifdefined\epTeXinputencoding % defined in e-pTeX (> TL2016)
  \epTeXinputencoding utf8    % ensure utf-8 encoding for platex
\fi

\documentclass[a4paper]{jsarticle}
\usepackage{doc}
\usepackage{array}
\usepackage{plextarray}
\GetFileInfo{plextarray.sty}
\title{Package \textsf{plextarray} \fileversion}
\author{Hironobu Yamashita}
\date{\filedate}
\begin{document}

\maketitle

\begin{center}
日本語文書は\texttt{plextarray-ja.pdf}です。
\end{center}

Package \textsf{array}, included in \textsf{latex-tools} bundle, is
incompatible with \textsf{plext} package for p\LaTeX.
This is because \textsf{plext} extends the environment \texttt{tabular}
and \texttt{array} to add an optional argument for direction specification.
This new package \textsf{plextarray} solves this incompatibility, and enables
coexistence of both extensions provided by \textsf{plext} and \textsf{array}.

Package \textsf{plext} extends \texttt{tabular} and \texttt{array}
environments by adding \texttt{<dir>} option, which specifies the
writing direction:
\begin{verbatim}
  \begin{tabular}<dir>[pos]{table spec} ... \end{tabular}
  \begin{array}<dir>[pos]{table spec} ... \end{array}
\end{verbatim}
This option permits one of the following three values.
If none of them is specified, the direction inside the environment
is same as that outside the enviromnent.
\begin{quote}
  \begin{description}
  \item[y] \emph{yoko} direction (horizontal writing)
  \item[t] \emph{tate} direction (vertical writing)
  \item[z] native direction of \TeX
\end{description}
\end{quote}

This package is part of \textsf{platex-tools} bundle:
\begin{verbatim}
  https://github.com/aminophen/platex-tools
\end{verbatim}

\end{document}
