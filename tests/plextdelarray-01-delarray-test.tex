\documentclass{jsarticle}
\usepackage{delarray}
\usepackage{plextdelarray} % 無理やりパッチ
\title{delarray + plextのテスト}
\author{Hironobu Yamashita}
\begin{document}
\maketitle
delarray.styが提供する機能で\verb+\left+と\verb+\right+の括弧を付ける
\[
  \begin{array}[t]\{{c}\}
    3 \\ 4 \\ 5
  \end{array}
  \begin{array}[c]\{{c}\}
    2 \\ 3 \\ 4
  \end{array}
  \begin{array}[b]\{{c}\}
    1 \\ 2 \\ 3
  \end{array}
\]
この機能(特に\verb+[t]+と\verb+[b]+)は、delarrayなしでは実現できない。
\[
\left\{
  \begin{array}[t]{c}
    3 \\ 4 \\ 5
  \end{array}
\right\}
\left\{
  \begin{array}[c]{c}
    2 \\ 3 \\ 4
  \end{array}
\right\}
\left\{
  \begin{array}[b]{c}
    1 \\ 2 \\ 3
  \end{array}
\right\}
\]
ただ、この拡張機能はplext.styの「組方向オプション」\verb+<y>+などと
衝突し、\verb+<>+の括弧をdelarray向けの「左右括弧」のつもりで使うと、
実際の処理ではplextの「組方向オプション」と誤認される恐れがある。
\iffalse
\[
  \begin{array}<{cc}>
    a_{11} & a_{12} \\
    a_{21} & a_{22}
  \end{array}
\]
\fi
% ==================================
% ! Missing delimiter (. inserted).
% ==================================
しかし、これは元のdelarray.styでも、「左右括弧」に\verb+[]+を使う
場合にはこれが「垂直位置指定」と誤認されて類似の問題が起きる。
\iffalse
\[
  \begin{array}[{cc}]
    a_{11} & a_{12} \\
    a_{21} & a_{22}
  \end{array}
\]
\fi
そこで、これらの挙動を含めて「仕様」と設定する。
特にplextdelarrayの場合は、delarrayの左右括弧に\verb+<>+でなく
\verb+\langle+と\verb+\rangle+を使えば問題は生じない。
\[
  \begin{array}\langle{cc}\rangle
    a_{11} & a_{12} \\
    a_{21} & a_{22}
  \end{array}
\]
こうすると、plext.styとdelarray.styが(一応)共存できるようになる。
\[
  \begin{array}<t>[t]\{{c}\}
    3 \\ 4 \\ 5
  \end{array}
  \begin{array}<t>[c]\{{c}\}
    2 \\ 3 \\ 4
  \end{array}
  \begin{array}<t>[b]\{{c}\}
    1 \\ 2 \\ 3
  \end{array}
\]
\[
\newcolumntype{L}{>{$}l<{$}}
f(x)=
  \begin{array}<t>\{{lL}.
    0 & if $x=0$ \\
    \sin(x)/x & otherwise
  \end{array}
\]
…必要なのだろうか? とはいえ、delarrayを使うと「plextの部分的な
縦書き機能(tabular環境を含む)までキャンセルされる」という従来の
挙動は嬉しく場合もあるだろう。無いよりはマシかもしれない。
\end{document}
