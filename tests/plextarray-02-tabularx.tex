% http://www.biwako.shiga-u.ac.jp/sensei/kumazawa/tex/tabularx.html
\documentclass{jsarticle}
\usepackage{tabularx} % 内部で array.sty を読む
\usepackage{plextarray} % これで array.sty にパッチ
% 通常は array/tabularx より後に plext を読むと有害
% だが plextarray を使えば大丈夫
\title{tabularx + plextのテスト}
\author{Hironobu Yamashita}
\begin{document}
\maketitle

幅を規定した表はtabularx.styを使う。ただし、tabularx.styは内部で
array.styを読み込むため、そのままではplext.styと共存しない。

plextarrayパッケージはtabular*環境にも対応している。

\begin{tabular*}<y>{10cm}{@{\extracolsep{\fill}}|c|c|c|r|}
  \hline
  ラベル1 & ラベル2 & ラベル3 & ラベル4 \\ \hline
  項目1  & 項目2 & 項目3 & 項目4 \\ \hline
\end{tabular*}

\medskip

\newcolumntype{C}{>{\centering\arraybackslash}X}
\newcolumntype{R}{>{\raggedright\arraybackslash}X}
\newcolumntype{L}{>{\raggedleft\arraybackslash}X}

したがって、plextarrayパッケージを使用すれば、plextパッケージを
読み込んでいても、tabularx環境を使えるようになる。

\begin{tabularx}{.75\textwidth}{|l|X|r|X|} \hline
  作者 & 冒頭 & 発表時期 & 備考 \\ \hline
  夏目漱石 & 吾輩は猫である。名前はまだ無い。 & 明治時代 &
  夏目漱石の処女作でもある\footnote{長編小説です。}。 \\ \hline
\end{tabularx}

\begin{tabularx}{9cm}{|c|R|c|L|}\hline
  \multicolumn{2}{|c|}{Multicolumn entry!} & THREE & FOUR \\ \hline
  夏目漱石 & 吾輩は猫である。名前はまだ無い。 & 長編小説 &
  何でも薄暗いじめじめした所でニャーニャー泣いていた
  事だけは記憶している。 \\ \hline
\end{tabularx}

\medskip

ただし、tabularx環境の組方向を変えることは想定していない。
\iffalse
% plextarray.sty 2018/09/18 v1.1b までは勘違いで
% 引数を「幅」→「組方向オプション」という逆順に指定できる状態だった。
%\begin{tabularx}{9cm}<t>{|c|R|c|L|}\hline
%  \multicolumn{2}{|c|}{Multicolumn entry!} & THREE & FOUR \\ \hline
%  夏目漱石 & 吾輩は猫である。名前はまだ無い。 & 長編小説 &
%  何でも薄暗いじめじめした所でニャーニャー泣いていた
%  事だけは記憶している。 \\ \hline
%\end{tabularx}
% この場合は「組方向変更が機能するが結果が変」だった。
% しかし、plextarray.sty 2018/09/20 v1.1c で plext.sty と互換な
% 「組方向オプション」→「幅」の順に変更された。
% tabularx 環境は第一引数を取って「幅」に与える仕様のため、
% 上記の変更に伴い組方向指定が禁止された。
\fi

\clearpage
\renewcommand{\arraystretch}{1.5}

通常のtabular環境だと
\begin{tabular}{|l|c|c|c|c|c|c|} \hline
   &       $a$  & $b$ & $c$ & $p$ &         $q$  & $h$ \\ \hline\hline
a) &        6   &  12 &     &     &              &     \\ \hline
b) &            &   4 &   6 &     &              &     \\ \hline
c) &            &     &  12 &     &              &  6  \\ \hline
d) &        6   &     &  12 &     &              &     \\ \hline
e) &        4   &     &     &   2 &              &     \\ \hline
f) &        6   &     &     &   8 &              &     \\ \hline
g) &            &   4 &     &     & $4/\sqrt{5}$ &     \\ \hline
h) &        8   &     &     &     &              &  6  \\ \hline
i) &            &     &  12 &   4 &              &     \\ \hline
j) & $\sqrt{5}$ &   6 &     &     &              &     \\ \hline
\end{tabular}

tabularx環境だと
\begin{tabularx}{100mm}{*{7}{|C}|} \hline
   &       $a$  & $b$ & $c$ & $p$ &         $q$  & $h$ \\ \hline\hline
a) &        6   &  12 &     &     &              &     \\ \hline
b) &            &   4 &   6 &     &              &     \\ \hline
c) &            &     &  12 &     &              &  6  \\ \hline
d) &        6   &     &  12 &     &              &     \\ \hline
e) &        4   &     &     &   2 &              &     \\ \hline
f) &        6   &     &     &   8 &              &     \\ \hline
g) &            &   4 &     &     & $4/\sqrt{5}$ &     \\ \hline
h) &        8   &     &     &     &              &  6  \\ \hline
i) &            &     &  12 &   4 &              &     \\ \hline
j) & $\sqrt{5}$ &   6 &     &     &              &     \\ \hline
\end{tabularx}

\end{document}
