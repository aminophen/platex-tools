%# -*- coding: utf-8 -*-
\ifx\epTeXinputencoding\undefined\else % defined in e-pTeX (> TL2016)
  \epTeXinputencoding utf8    % ensure utf-8 encoding for platex
\fi

\documentclass[a4paper]{jsarticle}
\usepackage{doc}
\usepackage{pxeveryshi}
\GetFileInfo{pxeveryshi.sty}
\title{Package \textsf{pxeveryshi} \fileversion}
\author{Takayuki Yato \& Hironobu Yamashita}
\date{\filedate}
\begin{document}

\maketitle

Package \textsf{everyshi}, written by Martin Schr\"oder,
cannot be used with Japanese classes for vertical writing
(tarticle.cls, etc). This behavior is not comfortable, since
\textsf{everyshi} is required by some other packages
(e.g. \textsf{tikz}) and causes an error:
\begin{verbatim}
  ! Incompatible direction list can't be unboxed.
\end{verbatim}
The package \textsf{pxeveryshi} provides a patch for \textsf{everyshi}
to work with both horizontal and vertical writing.

With \pLaTeXe~2020-10-01 or later, this package does
nothing because the \pLaTeXe\ itself handles the situation.
Therefore, you will not need this package at all.

This package is part of \textsf{platex-tools} bundle:
\begin{verbatim}
  https://github.com/aminophen/platex-tools
\end{verbatim}

\bigskip

Martin Schr\"oder氏による\textsf{everyshi}パッケージは、
p\LaTeX の縦組クラス(tarticle.clsなど)で使うとエラーが出てしまいます。
この挙動は、たとえば\textsf{tikz}のような内部で\textsf{everyshi}パッケージ
を必要とするパッケージを縦組で使用する際に問題になります。
この\textsf{pxeveryshi}パッケージを読み込むことで、縦組クラスでも
\textsf{everyshi}パッケージ(およびこれに依存するパッケージ)が
使えるようになります。

\pLaTeXe~2020-10-01以降では、同等の機能が\pLaTeXe{}カーネルに
存在するため、\textsf{pxeveryshi}パッケージは何もしません。

\section{使いかた}

使いかたは、\textsf{everyshi}パッケージの代わりに、あるいは
\textsf{everyshi}パッケージに追加して、\textsf{pxeveryshi}パッケージを
読みこむだけです。使用例:
\begin{verbatim}
  \documentclass{tarticle}
  \usepackage[dvipdfmx]{graphicx}
  \usepackage{tikz}
  \usepackage{pxeveryshi}
  \begin{document}
  \begin{tikzpicture}
    \draw (0,0) -- (1,2);
  \end{tikzpicture}
  \end{document}
\end{verbatim}

\section*{References}

\begin{itemize}
\item tikz を縦書きで\\
  \texttt{https://oku.edu.mie-u.ac.jp/~okumura/texfaq/qa/56760.html}
\item Re: tikz を縦書きで\\
  \texttt{https://oku.edu.mie-u.ac.jp/~okumura/texfaq/qa/56761.html}
\end{itemize}

\end{document}
